\documentclass{beamer}
\usepackage{
    graphicx, float, caption,
    amsmath,
    textpos, blindtext, scrextend,
    apacite
}
\usepackage[font=small, labelfont=bf]{subcaption}

%% MARK: Beamer Setup
\usetheme{Darmstadt}
% Remove the useless and unsightly black bar at the top of the slides
\setbeamertemplate{headline}{}
% Remove the navigation thing at the bottom of the slides.
\setbeamertemplate{navigation symbols}{}
% Colors for Auburn University from their marketing materials
% http://www.ocm.auburn.edu/graphicservices/colors.html
\definecolor{auburn_orange}{RGB}{232, 119, 34}
\definecolor{auburn_blue}{RGB}{12, 35, 64}
% Set the color of the slide headers to auburn blue
\setbeamercolor{structure}{fg=auburn_blue}

%% MARK: Frontmatter
\title{Playing Super Mario Bros. with Deep-$Q$ Learning}
\author{Christian Kauten, Chaowei Zhang}
\institute{Auburn University}
\date{April 23, 2018}

%% MARK: Slides
\begin{document}
% title slide
\frame{\titlepage}



\begin{frame}{Problem}
\begin{minipage}{\textwidth}
%
\begin{minipage}{0.5\textwidth}
\begin{itemize}
    \item{Atari 2600}
        \begin{itemize}
        \item{Breakout, Enduro, Pong, Seaquest, Space Invaders}
        \end{itemize}
    \item{NES}
        \begin{itemize}
        \item{Super Mario Bros.}
        \end{itemize}
\end{itemize}
\end{minipage}
%
\hfill
%
\begin{minipage}[t]{0.5\textwidth}
\centering
\includegraphics[width=0.75\textwidth]{img/smb} \\
$\Downarrow$ \\
$f : \mathcal{S} \to \mathcal{A}$ \\
$\Downarrow$ \\
\includegraphics[width=0.85\textwidth]{img/nes}
\end{minipage}
%
\end{minipage}
\end{frame}



\begin{frame}{Dataset}
\begin{minipage}{\textwidth}
%
\begin{minipage}{0.6\textwidth}
\begin{itemize}
    \item{Defined by the ROM for each game}
    \item{Experience Replay of $1000000$ experiences}
    \item{Generated using "bootstrapping"}
    \item{Networks train on random mini-batches of $32$ experiences every $4$ decisions}
\end{itemize}
\end{minipage}
%
\hfill
%
\begin{minipage}{0.35\textwidth}
\centering
\includegraphics[width=\textwidth]{img/brain}
\end{minipage}
%
\end{minipage}
\end{frame}



\begin{frame}{Methods}
\framesubtitle{Double Deep-Q Learning}

\begin{equation}
y = r + (1 - d) \gamma \max_{a' \in \mathcal{A}} Q(s', a', \theta_{target})
\end{equation}

\begin{equation}
\hat{y} = Q(s, a, \theta)
\end{equation}

\begin{equation}
L(\theta) =
\mathbb{E}_{(s, a, r, d, s') \sim U(D)} \bigg[ L_{\delta}(y, \hat{y}) \bigg]
\label{eqn:deep-q-alg}
\end{equation}

\begin{equation}
L_{\delta}(y, \hat{y}) = \begin{cases}
      \frac{1}{2} (y - \hat{y})^2                & |y - \hat{y}| \leq \delta \\
      \delta |y - \hat{y}| - \frac{1}{2}\delta^2 & \textbf{otherwise} \\
\end{cases}
\label{eqn:huber}
\end{equation}

\end{frame}



\begin{frame}{Methods}
\framesubtitle{Dueling Deep-Q Network}
\begin{figure}
\includegraphics[width=0.9\textwidth]{img/dueling-deep-q}
\caption*{Deep-Q Network (top) and Dueling Deep-Q Network (Bottom)}
\end{figure}
\end{frame}



\begin{frame}{Results}
TODO
\end{frame}

\begin{frame}{Demo}
TODO
\end{frame}

\begin{frame}{Conclusions}
TODO
\end{frame}

\begin{frame}{Q\&A}
TODO
\end{frame}

\end{document}
