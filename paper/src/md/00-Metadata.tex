\title{Playing Super Mario Bros. with Deep Reinforcement Learning}

\author{
	James C. Kauten \\
	Department of Software Engineering \\
	Auburn University \\
	Auburn, AL 36832 \\
	\texttt{jck0022@auburn.edu} \\
	\And
	Chaowei Zhang \\
	Department of Software Engineering \\
	Auburn University \\
	Auburn, AL 36832 \\
	\texttt{czz0032@auburn.edu} \\
}

\maketitle

\begin{abstract}

Video games are more than just a fun way to pass the time. From a computer
science perspective, a video game represents a task of mapping states in
pixel space to actions in a discrete action space. As a task with exponential
growth, no deterministic algorithm can produce solutions to it in realistic
time. As such, contemporary research presents methods of approximating this
mapping to provide confident estimates with an algorithm of polynomial time
complexity. This work studies one such algorithm, Deep $Q$-learning, and its
ability to generalize to a new game, Super Mario Bros.

\end{abstract}
